\section{Testing} % (fold)
\label{cha:Testing}
\texttt{TODO: Problem mit Berechnungsdauer!}
Um nun verschiedene Puzzleinstanzen und deren l"osbarkeit anhand des Codes zu testen, werden verschiedene Startzust"ande und zu Testzwecken verschiedene Endzust"ande definiert. Anschlie"send werden alle Startzust"ande auf l"osbarkeit des \enquote{normalen} Endzustandes getestet.\\
Um eine verbose Ausgabe zu erm"oglichen, bei der noch die verschiedenen zuvor berechneten Werte ausgegeben werden k"onnen, kann der Funktion, welche die Puzzleinstanzen auf ihre L"osbarkeit pr"uft noch ein entsprechender verbose Flag mitgegeben werden. Somit kann anhand schon bekannten l"osbaren Instanzen "uberpr"uft werden, ob die Implementierung vollst"andig ist.\\
Der Output dieses Testes befindet sich im dazugeh"origen \textcolor{violet}{\href{https://github.com/stubifox/ai-termpaper/blob/main/code/15-solvable-v1.ipynb}{Jupyter-Notebook}}.\\
Diese Puzzle werden dann bspw. in das schon aus der Vorlesung bekannte \textcolor{violet}{\href{https://github.com/karlstroetmann/Artificial-Intelligence/blob/master/Python/1\%20Search/Iterative-Deepening-A-Star-Search.ipynb}{Jupyter-Notebook}} zum l"osen der Instanzen gegeben, so dass geschaut werden kann, ob diese in angemessener Zeit l"osbar sind und wieviele Z"uge f"ur die jeweiliege L"osung gebraucht werden.
% chapter Testing (end)