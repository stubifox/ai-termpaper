\section{Testing} % (fold)
\label{cha:Testing}
Um nun verschiedene Puzzleinstanzen und deren L"osbarkeit anhand des Codes zu testen, werden verschiedene Startzust"ande und zu Testzwecken verschiedene Endzust"ande definiert. Anschlie"send werden alle Startzust"ande auf L"osbarkeit des \enquote{normalen} Endzustandes getestet.\\
Um eine verbose Ausgabe zu erm"oglichen, bei der noch die verschiedenen zuvor berechneten Werte ausgegeben werden k"onnen, kann der Funktion, welche die Puzzleinstanzen auf ihre L"osbarkeit pr"uft, noch eine entsprechende verbose Flag mitgegeben werden. Somit kann anhand von bekannten l"osbaren Instanzen "uberpr"uft werden, ob die Implementierung vollst"andig ist.\\
Der Output dieses Testes befindet sich im dazugeh"origen \textcolor{violet}{\href{https://github.com/stubifox/ai-termpaper/blob/main/code/15-solvable-v1.ipynb}{Jupyter-Notebook}}.\\
Diese Puzzle werden dann bspw. in das schon aus der Vorlesung bekannte \textcolor{violet}{\href{https://github.com/karlstroetmann/Artificial-Intelligence/blob/master/Python/1\%20Search/Iterative-Deepening-A-Star-Search.ipynb}{Jupyter-Notebook}} zum L"osen der Instanzen gegeben, so dass geschaut werden kann, ob diese in angemessener Zeit l"osbar sind, und wieviele Z"uge f"ur die jeweilige L"osung gebraucht werden. \\
Ein Screenshot des Outputs findet sich im Anhang unter \ref{app:fig:testing-output}.
Das Problem hierbei ist, dass nicht jedes als l"osbar einkategorisierte Puzzle auch in angemessener Zeit mit Hilfe des \textit{IDA*} oder \textit{A*} Algorithmus gel"ost werden kann. So ist es durchaus vorgekommen, dass die Schritte zur L"osbarkeit eines Puzzles mehrere Stunden dauern (vgl. \ref{app:fig:1h}), da vorher nicht bekannt ist, wieviele Schritte zur L"osung f"uhren.\\
Somit wei"s man initial nicht, ob das Puzzle richtig einklassifiziert wurde.
% chapter Testing (end)