\section{Umsetzung} % (fold)
\label{cha:Umsetzung}
\texttt{Der vollst"andige Quellcode ist im Anhang unter \ref{code:validate-15-puzzle:py} zu finden.}\\\\
Zur Umsetzung wird zu erst eine Datenstruktur definiert, in welcher die Instanzen des 15-Puzzels ausgewertet werden. Hierf"ur werden Tupel von Tupeln verwendet.
Anschlie"send ist die gernerelle Idee, wie auch schon in \texttt{kais abschnitt?} vorgestellt zu schauen, ob die Anzahl der Permutationen bei der Datenstruktur als $1$-Dimensionale Liste und der Abstand des Blank-Feldes vom Start- zum Zielzustand die gleiche Parit"at besitzen.
Ist diese gleich, so ist das Puzzel l"osbar, vice versa unl"osbar.\\
Spannend ist hier allerdings, dass dies kein Indikator f"ur die Komplexit"at der L"osbarkeit gibt. So kann ein l"osbares Puzzel mit den zur Verf"ugung stehenden Algorithmen wie \textit{A*} oder \textit{IDA*} nicht in angemessener Zeit gel"ost werden.\\
Das weitere Vorgehen der Implementierung ist weitesgehend die Bearbeitung von kleinen Teilproblemen, welche sich aus dem eben genannten Vorgehen ergeben.\\
So betrachten wir zun"achst die Anzahl der Permutationen, welche sich in einer Puzzelinstanz befinden. Um diese zu berechnen wird als Datenstruktur eine Liste mit der Dimension 1 ben"otigt.
Anschlie"send muss die Liste durch tauschen der Elemente sortiert werden.
Praktischerweise ist der L"osungszustand des Puzzels so definiert, dass der Index innerhalb der Liste und der Wert des zugeh"origen Elementes identisch sind. Somit kann f"ur jeden Index der entsprechende Wert innerhalb des Arrays gesucht werden, sodass eine minimale Anzahl von \textit{swaps} durchgef"uhrt werden. \\Die Anzahl der get"atigten \textit{swaps} liefert uns dann die Anzahl der vorhandenen Permutationen innerhalb der Puzzelinstanz.\\
Die Implementierung ist dabei wie folgt:
\begin{minted}[linenos,breaklines,breakanywhere]{python}
def get_inversion_count(Puzzle: tuple) -> int:
    working_1d_puzzle = to_1d(Puzzle)
    count = 0
    for i in range(len(working_1d_puzzle)):
        if working_1d_puzzle[i] != i:
            count += 1
            swap(i, find_tile_1d(i, working_1d_puzzle), working_1d_puzzle)
    return count
\end{minted}
wobei die Funktion \textbf{to\_1d} definiert ist durch
\vspace{.25cm}
\begin{minted}[breaklines,breakanywhere]{python}
to_1d = lambda Puzzle: [elem for tupl in Puzzle for elem in tupl]
\end{minted}
\vspace{.25cm}
und die Funktion \textbf{swap} durch
\vspace{.25cm}
\begin{minted}[breaklines,breakanywhere]{python}
def swap(idxA, idxB, Puzzle_1d):
    Puzzle_1d[idxA], Puzzle_1d[idxB] = Puzzle_1d[idxB], Puzzle_1d[idxA]
\end{minted}
\vspace{.25cm}
definiert ist.
Die Funktion \textbf{find\_tile\_1d} gibt den Index einer 1-Dimensionalen Liste zur"uck, an der das entsprechende Element zu finden ist. Diese ist auch im Anhanhg unter \ref{code:validate-15-puzzle:py} zu finden.\\
Nun muss als n"achstes die Distanz des Blank-Feldes vom Startzustand zum Zielszustand berechnet werden. Hierbei wird die $x,y$ Position des Blank-Feldes im Startzustand gesucht und anschlie"send "ahnlich wie bei der Manhattan Distanz die Absolute Differenz beider Zust"ande addiert. Da die $x,y$ Position des Blank-Feldes im Endzustand immer $0,0$ ist, muss diese nicht berechnet werden, wird in der Funktion allerdings auch gesucht, sodass bei einer Erweiterung noch der Endzustand variieren kann.
Sei $\alpha,\beta$ nun die Position des Blank-Feldes der Matrix $A$ und $\gamma,\epsilon$ die Position des Blank-Feldes der Matrix $B$, so ist der Abstand beider Blank-Felder durch \\
\begin{center}
    $\left | \alpha - \gamma \right | + \left | \beta - \epsilon \right |$
\end{center}
definiert.
Die dazugeh"orige Funktion \textbf{manhattan} ist auch im Anhang unter \ref{code:validate-15-puzzle:py} zu finden.\\
Zuallerletzt muss nur noch geschaut werden, ob das Produkt der Distanz und die Anzahl der Permutationen im Startzustand die gleiche Parit"at besitzen. Damit wissen wir nun, ob diese Instanz l"osbar ist.
% \inputminted[linenos,breaklines,breakanywhere]{python}{../code/15-solvable-v1.py}
% chapter Umsetzung (end)