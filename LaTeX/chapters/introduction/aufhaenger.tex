\section{Geschichte} % (fold)
\label{cha:Geschichte}
Die erste bekannte Erscheinung des heute so bekannten Puzzles ist in den 1870'er Jahren in Amerika aufgetreten. Bisher ist man davon ausgegangen, dass der Erfinder des Puzzles der Amerikaner Sam Loyd ist, jedoch ist nach einer Untersuchung von Jerry Slocum und Dieter Gebhardt Sam Loyd gar nicht der echte Erfinder des 15-Puzzles \autocite{anchor-puzzle:book, the-15-puzzle:online}. Demnach hat Sam Loyd die Idee nur gut vermarktet und sich selbst dadurch in die "Offentlichkeit gestellt \autocite{wiki-15-puzzle:online}.\\
Nichtsdestotrotz hat Sam Loyd es geschafft, die Aufmerksamkeit der "Offentlichkeit auf das Puzzle zu lenken und somit das Interesse vieler zu wecken.\\
Das Ziel des Puzzles ist es 15 Puzzlesteine numeriert von 1-15 auf einer 4 x 4 Ebene durch Verschiebung in seine Urspr"ungliche sortierte Form zur"uckzubringen (vgl. Abb.\ref{fig:puzzle-end}).
\begin{figure}[H]
    \centering
    \includegraphics[width=.3\textwidth,keepaspectratio]{img/Fifteen_puzzle.png}
    \captionsetup{format=hang}
    \caption[Zielzustand des 15 Puzzles]{\label{fig:puzzle-end}Zielzustand des 15 Puzzles}
\end{figure}
Der schnelle Ruhm des Puzzles ergab sich nun daraus, dass Sam Loyd ein Preisgeld von \$$1000$ f"ur denjenigen ausschrieb, der sein Puzzle l"osen kann. Darauf folgte ein "offentlicher Ansturm auf das Puzzle, welches als \enquote{14-15 Puzzle} bekannt wurde, da lediglich die 14 und 15 vertauscht wurden. (Vgl. Abb.\ref{fig:puzzle-illustration})
\begin{figure}[H]
    \centering
    \includegraphics[width=.6\textwidth,keepaspectratio]{img/sam-loyd-puzzle-illustration.jpg}
    \captionsetup{format=hang}
    \caption[Illustration des 14-15 Puzzles von Sam Loyd]{\label{fig:puzzle-illustration}Illustration des 14-15 Puzzles von Sam Loyd (1914) \\Quelle: \cite[pp. 234-235]{loyd-cyclopedia:book}}
\end{figure}
Allerdings zeigte sich im Nachhinein, dass das Puzzle unl"osbar ist, da es eine Transformation von einer geraden zu einer ungeraden Permutation erfordert \autocite{wiki-15-puzzle:online}. Die Unl"osbarkeit des Puzzles wurde erstmals von Wm. Woolsey Johnson und William E. Story im Jahre 1879 bei einer Ver"offentlichung des American Journal of Mathematics bewiesen. \autocite{ajom-notes-15-puzzle:article}
Somit gewann keiner das Preisgeld von \$$1000$. Das 15-Puzzle wurde daraufhin auch bei einer Illustration mit dem Titel \enquote{The Great Presidential Puzzle} der US-Pr"asidentschaftswahl 1880 referenziert. Zu finden ist dies in der United States Library of Congress's Prints and Photographs division unter der Digitalen ID \textcolor{violet}{\href{https://www.loc.gov/rr/print/}{ppmsca. 15782}} \autocite{presidental-puzzle:online}.