\section{Aufgabenstellung} % (fold)
\label{cha:Aufgabenstellung}
H"atte man damals bewiesen, dass das Puzzel aus der Einleitung von Sam Loyd nicht l"osbar ist, w"are wohl vielen Menschen n"achtelange Verzweiflung erspart geblieben.\\
Nun stellt sich Frage wann eine Instanz des Puzzels l"osbar ist. Allgemein gesprochen ist eine Instanz eines 15 Puzzels dann l"osbar, falls es eine Sequenz von zul"assigen Z"ugen gibt, welche vom Startzustand zum Zielzustand f"uhren.\\
Diese Annahme gilt als Zutreffend f"ur genau die h"alfte aller m"oglichen $16! \approx 2 \cdot 10^{13}$ Puzzel Kombinationen. \autocite{sliding-piece-puzzels:book,solving-15-puzzle-lvi:article}
Das Puzzel Problem ist ein klassischer Anwendungsfall wenn es um Modellierung von Algorithmen mit Heuristiken geht. "Ublicherweise werden Algorithmen wie \textit{A*} oder \textit{IDA*} zur L"osung solcher Heurisiken genutzt.\autocite{wiki-15-puzzle:online,solving-15-puzzle-lvi:article, depth-first-id:article}
Im Rahmen dieser Arbeit geht es nun darum bei gegebener Puzzelinstanz vorherzusagen, ob diese als L"osbar oder Unl"osbar gilt.

\texttt{Abgrenzung + Vorauswahl (was genau machen wir) => verweis auf stroeti: Wir beschaeftigen uns nicht, referenzierung auf skriptstelle, an der xy zu finden ist.}

% chapter Aufgabenstellung (end)