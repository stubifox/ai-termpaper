\section{Aufgabenstellung und Abgrenzung} % (fold)
\label{cha:Aufgabenstellung}
H"atte man damals bewiesen, dass das Puzzle aus der Einleitung von Sam Loyd nicht l"osbar ist, w"are wohl vielen Menschen n"achtelange Verzweiflung erspart geblieben.\\
Nun stellt sich die Frage, wann eine Instanz des Puzzles l"osbar ist. Allgemein gesprochen ist eine Instanz eines 15 Puzzles dann l"osbar, falls es eine Sequenz von zul"assigen Z"ugen gibt, welche vom Startzustand zum Zielzustand f"uhren.\\
Diese Annahme gilt als zutreffend f"ur genau die H"alfte aller m"oglichen $16! \approx 2 \cdot 10^{13}$ Puzzle Kombinationen \autocite{sliding-piece-puzzels:book,solving-15-puzzle-lvi:article}.
Das Puzzle Problem ist ein klassischer Anwendungsfall, wenn es um Modellierung von Algorithmen mit Heuristiken geht. "Ublicherweise werden Algorithmen wie \textit{A*} oder \textit{IDA*} zur L"osung solcher Heuristiken genutzt \autocite{wiki-15-puzzle:online,solving-15-puzzle-lvi:article, depth-first-id:article}.
Im Rahmen dieser Arbeit geht es nun darum, bei gegebener Puzzleinstanz vorherzusagen, ob diese als l"osbar oder unl"osbar gilt.
Was bei dieser Arbeit nicht betrachtet wird, sind die verschiedenen Algorithmen wie \textit{A*} oder auch \textit{IDA*}. Diese werden ausf"uhrlich im zugeh"origen \textcolor{violet}{\href{https://github.com/karlstroetmann/Artificial-Intelligence/blob/master/Lecture-Notes/artificial-intelligence.pdf}{Vorlesungsskript}} \autocite{github-stroetmann:online} in den Sektionen \textit{\textbf{2.7} A* Search} und \textit{\textbf{2.10} A*-IDA* Search} erl"autert.
Lediglich werden diese genutzt, um die als l"osbar identifizierten Puzzles anschlie"send auch zu l"osen.



% chapter Aufgabenstellung (end)