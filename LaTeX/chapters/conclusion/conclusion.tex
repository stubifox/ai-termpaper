% !TEX root =  master.tex
\chapter{Zusammenfassung}
Diese Arbeit zeigt, dass Instanzen des 15-Puzzle programmatisch auf ihre L"osbarkeit "uberpr"uft werden k"onnen, indem sie als Permutationen in einer sortierten Liste betrachtet werden.
Der daraus resultierende von Bradlow vorgestellte Algorithmus ist leicht zu verstehen und l"asst sich ohne Umwege in einem Python-Modul abbilden. Dieses kann als Filter genutzt werden, um sicherzustellen, dass es eine L"osung gibt, die die in der Vorlesung behandeltelten Suchalgorithmen \cite[vgl.][Kap. 2]{github-stroetmann:online} finden  k"onnen.\\
Die "Uberpr"ufung durch einen Suchalgorithmus, der alle m"oglichen Zust"ande durcharbeitet ist mit einem h"oherem Zeitaufwand verbunden, als es der Filter ben"otigt.
So kann fr"uhzeitig erkannt werden, dass ein Puzzle nicht l"osbar ist und weder Suchalgorithmen noch Menschen wie in der Einleitung \ref{cha:Geschichte} m"ussen n"achtelang versuchen eine unl"osbare Aufgabe zu l"osen.
\WNL
Zuk"unftig k"onnte das Modul so erweitert werden, dass beliebige Endzust"ande m"oglich sind, wie bspw. in \ref{code:validate-15-puzzle:py}(Z. 52 - Z. 79) definiert sind. Aktulel wird nur auf die L"osbarkeit mit dem von Herrn Stroetmann definierten Endzustand gepr"uft