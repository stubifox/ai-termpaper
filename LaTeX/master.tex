%%%%%%%%%%%%%%%%%%%%%%%%%%%%%%%%%%%%%%%%%%%%%%%%%%%%%%%%%%
%   Autoren:
%   Prof. Dr. Bernhard Drabant
%   Prof. Dr. Dennis Pfisterer
%   Prof. Dr. Julian Reichwald
%%%%%%%%%%%%%%%%%%%%%%%%%%%%%%%%%%%%%%%%%%%%%%%%%%%%%%%%%%

%%%%%%%%%%%%%%%%%%%%%%%%%%%%%%%%%%%%%%%%%%%%%%%%%%%%%%%%%%
%	ANLEITUNG: 
%   1. Ersetzen Sie firmenlogo.jpg im Verzeichnis img
%   2. Passen Sie alle Stellen im Dokument an, die mit 
%      @stud 
%      markiert sind
%%%%%%%%%%%%%%%%%%%%%%%%%%%%%%%%%%%%%%%%%%%%%%%%%%%%%%%%%%

%%%%%%%%%%%%%%%%%%%%%%%%%%%%%%%%%%%%%%%%%%%%%%%%%%%%%%%%%%
%	ACHTUNG: 
%   Für das Erstellen des Literaturverzeichnisses wird das 
%   modernere Paket biblatex in Kombination mit biber 
%   verwendet - nicht mehr das ältere Paket BibTex!
%
%   Bitte stellen Sie Ihre TeX-Umgebung entsprechend ein (z.B. TeXStudio): 
%   Einstellungen --> Erzeugen --> Standard Bibliographieprogramm: biber
%%%%%%%%%%%%%%%%%%%%%%%%%%%%%%%%%%%%%%%%%%%%%%%%%%%%%%%%%%

\documentclass[fontsize=12pt,BCOR=5mm,DIV=12,parskip=half,
               listof=entryprefix,paper=a4,toc=bibliography,toc=listof,pointlessnumbers]{scrreprt}
               
\makeindex

%% Elementare Pakete, Konfigurationen und Definitionen werden geladen (gegebenenfalls anpassen)
\input{config}

%%
%% @stud
%%
%% PERSÖNLICHE ANGABEN (BITTE VOLLSTÄNDIG EINGEBEN zwischen den Klammern: {...})
%%
\ArtDerArbeit{Seminar} % "Bachelor" oder "Projekt" wählen
\TitelDerArbeit{Analyse der Lösbarkeit von Instanzen des 15-Puzzzles mit Implementierung}
\AutorDerArbeit{Kai Fischer, Max Stubenbord}
\Kurs{TINF18AI1}
\Studienrichtung{angewandte Informatik}
\Matrikelnummer{3683691, 5379506}
\Studiengangsleiter{Prof. Dr. Holger Hofmann}
\WissBetreuer{Prof. Dr. Karl Stroetmann}
\Bearbeitungszeitraum{24.03.2021 -- 11.06.2021}
\Abgabedatum{11.06.2021}

%%
%% @stud
%%
%% BIBLIOGRAPHY (@stud: Bibliographie-Stil wählen - Position und Indizierung)
%%  Auswahl zwischen: 
%%   NUMERIC Style
%%   IEEE Style
%%   ALPHABETIC Style
%%   HARVARD Style
%%   CHICAGO Style 
%%   (oder eigenen zulässigen Stil wählen) 
%%
%%%%%%%%%%%%%
%% Zitierstil
%%%%%%%%%%%%%
% NUMERIC Style - e. g. [12]
\newcommand{\indextype}{numeric} 
%
% IEEE Style - numeric kind of style 
%\newcommand{\indextype}{ieee} 
%
% ALPHABETIC Style - e. g. [AB12]
%\newcommand{\indextype}{alphabetic} 
%
% HARVARD Style 
%\newcommand{\indextype}{apa} 
%
% CHICAGO Style 
%\newcommand{\indextype}{authoryear}
%
%%%%%%%%%%%%%%%%%%%%%%
%% Position des Zitats
%%%%%%%%%%%%%%%%%%%%%%
\newcommand{\position}{inline} 
%
% (!!) FOOTNOTE POSITION NOT RECOMMENDED IN MINT DOMAIN:
%\newcommand{\position}{footnote}

%% Final: Setzen des Zitierstils und der Zitatposition
\usepackage[backend=bibtex, autocite=\position, style=\indextype]{biblatex} 	
\settingBibFootnoteCite

%%
%% Definitionen und Commands
%%
\newcommand{\abs}{\par\vskip 0.2cm\goodbreak\noindent}
\newcommand{\nl}{\par\noindent}
\newcommand{\mcl}[1]{\mathcal{#1}}
\newcommand{\nowrite}[1]{}
\newcommand{\NN}{{\mathbb N}}
\newcommand{\imagedir}{img}
\newcommand{\afz}[1]{\glqq#1\grqq}% used as shortcut for the Germanquotationstyle
\newcommand{\WNL}{\\[1ex]} %used for the spacy newlines inspired by MS Word
\newcommand{\MaxLink}[2]{\textcolor{violet}{\href{#1}{#2}}}% used for hypertext links styled like Stubifox

\makeindex

\begin{document}


\setTitlepage

%%%%%%%%%%%%%%%%%%%%%%%%%%%%%%%%%%%%%%%%%%%%%%%%%%%%%%%%%%%%%%%%%%%%%%%%%%%%%%%%%%%%%%%%%%
% KAPITEL UND ANHÄNGE
%
% @stud:
%   - nicht benötigte: auskommentieren/löschen
%   - neue: bei Bedarf hinzufügen mittels input-Kommando an entsprechender Stelle einfügen
%%%%%%%%%%%%%%%%%%%%%%%%%%%%%%%%%%%%%%%%%%%%%%%%%%%%%%%%%%%%%%%%%%%%%%%%%%%%%%%%%%%%%%%%%%

%%%%%%%%%%%%%%%%%%%%%%%%%%%%%%%%%%%
% EHRENWÖRTLICHE ERKLÄRUNG
%
% @stud: ewerkl.tex bearbeiten
%
\input{meta/ewerkl}
%%%%%%%%%%%%%%%%%%%%%%%%%%%%%%%%%%%

%%%%%%%%%%%%%%%%%%%%%%%%%%%%%%%%%%%
% SPERRVERMERK
%
% @stud: nondisclosurenotice.tex bearbeiten
%
%\input{meta/nondisclosurenotice}
%%%%%%%%%%%%%%%%%%%%%%%%%%%%%%%%%%%

%%%%%%%%%%%%%%%%%%%%%%%%%%%%%%%%%%%
%	KURZFASSUNG
%
% @stud: acknowledge.tex bearbeiten
%
% !TEX root =  master.tex
\chapter*{Danksagung}

Ein besonderer Dank geht an Prof. Dr. Stroetmann f"ur die Erm"oglichung einer alternativen Pr"ufungsleistung. Wie wir w"ahrend der Corona-Zeit festgestellt haben, ist ein solches Engagement nicht selbstverst"andlich.\\
Abschlie"send wollen wir uns auch bei dem Kurs TINF18AI1 bedanken f"ur die faire und strukturierte Verteilung der Themen.



%%%%%%%%%%%%%%%%%%%%%%%%%%%%%%%%%%%

%%%%%%%%%%%%%%%%%%%%%%%%%%%%%%%%%%%
%	KURZFASSUNG
%
% @stud: abstract.tex bearbeiten
%
% !TEX root =  master.tex
\chapter*{Abstract}

The 15-puzzle first gained public attention in the 1870s when Sam Loyd promised $\$1000$ in prize money to anyone who could solve his puzzle, which became known as the 14-15 puzzle.\\
It was not until nearly a decade later that authors of the American Journal of Mathematics published a proof that the puzzle is unsolvable.
Today, the 15-puzzle is a classic use case when it comes to modelling algorithms with heuristics, such as \textit{A*} or \textit{IDA*}.\\
The main purpose of this work is to outline and implement the algorithm for checking the solvability of an instance of the 15-puzzle.
This algorithm is based on Bradlow's approach where a puzzle is considered solvable, if the parity of the transposition count and the manhattan distance of the blank field match.\\The In the process written source-code can be found within the appendix. To test this implementation, an evaluation of specific test-cases is conducted.\\
Altough the algorithm specifies the solvability for a given puzzle instance, it remains uncertain how many steps are required to obtain this solution. Hence the algorithm provides no indicator for computing time.

%%%%%%%%%%%%%%%%%%%%%%%%%%%%%%%%%%%

%%%%%%%%%%%%%%%%%%%%%%%%%%%%%%%%%%%
% VERZEICHNISSE
%
% @stud: ggf. nicht benötigte Verzeichnisse auskommentieren/löschen in Def. von \settingLists in config.tex
%
\settingLists
%%%%%%%%%%%%%%%%%%%%%%%%%%%%%%%%%%%

%%%%%%%%%%%%%%%%%%%%%%%%%%%%%%%%%%%

\initializeText
\onehalfspacing

%%%%%%%%%%%%%%%%%%%%%%%%%%%%%%%%%%%
% KAPITEL
%
% @stud: einzelne Kapitel bearbeiten und eigene Kapitel hier einfügen
%
% Einleitung
% !TEX root =  master.tex
\chapter{Einleitung}
\section{Geschichte} % (fold)
\label{cha:Geschichte}

% chapter Geschichte (end)
\section{Aufgabenstellung} % (fold)
\label{cha:Aufgabenstellung}

% chapter Aufgabenstellung (end)
\section{Vorgehen} % (fold)
\label{cha:Vorgehen}
Das Vorgehen der Arbeit ist dabei wiefolgt:
% chapter Vorgehen (end)



% mehrere Grundlagen- und Forschungs-Kapitel
% !TEX root =  master.tex
\chapter{Grundlagen}
In diesem Kapitel werden die theoretischen Grundlagen und Gedanken für die Umsetzung im nächsten Kapitel \ref{chap:Implementation} vorgestellt. Die Lösung und das Vorgehen basieren maßgeblich auf dem Beitrag \afz{\MaxLink{https://www.youtube.com/watch?v=YI1WqYKHi78}{Why is this Puzzle Impossible? - Numberphile}} von Herrn Steven Bradlow zur Lösbarkeit des \afz{14-15 puzzles} aus der Einleitung auf dem Youtube Kanal \afz{Numberphile} \autocite{Unsolvable-14-15-Numberphile-YT:online}.%
%
\section{Puzzle zu Listen wandeln} % (fold)
\label{sec:PuzzleToList}
Der Lösungsansatz aus \autocite{Unsolvable-14-15-Numberphile-YT:online} basiert auf Permutationen. Um 4x4 Puzzle besser auf Permutationen untersuchen zu können, werden die Puzzle als Listen von Zahlen dargestellt. Dazu werden die Inhalte der Zellen des Puzzles Zeilenweise hintereinander in eine Liste geschrieben. Die Leerstelle, auch als \afz{blank} beschrieben, wird dabei als Zahl \afz{0} interpretiert.
Der Zustand des Puzzles aus Abb.\ref{fig:Perm_puzzle_start_Pic}
\begin{figure}[H]
	\centering
	\includegraphics[width=.5\textwidth,keepaspectratio]{img/Start_Puzzle2.png}
	\captionsetup{format=hang}
	\caption{Beispiel Zustand eines 4x4-Puzzles \label{fig:Perm_puzzle_start_Pic}}
\end{figure}
\begin{minipage}{\linewidth}
	wird als Liste aus Zahlen wie folgt dargestellt:
	\begin{center}
		$State = \{0,1,2,3,4,5,6,8,14,7,11,10,9,15,12,13\}$
	\end{center}
\end{minipage}\WNL%
Wichtig ist bei der Betrachtung der lösbaren Puzzle und des Vorgehen der Lösung aus \autocite{Unsolvable-14-15-Numberphile-YT:online} aber auch anderer Verfügbarer Quellen wie \autocite{solving-15-puzzle-lvi:article,geeksforgeeks:online,archer-15-puzzle:article}. Die Bezeichnung der Leerstelle oder die Art der Konvertierung eines Puzzles zu einer Liste variiert. Die meisten Lösungen sehen den Zielzustand aus Abb.\ref{fig:Perm_puzzle_end_allOther} vor, wobei die Leerstelle dann die Nummer \afz{16} trägt. Um mit den Darstellungen von Herrn Stroetmann aus dem Vorlesungsskript \autocite{github-stroetmann:online} übereinzustimmen, wird der Zielzustand aus Abb.\ref{fig:Perm_puzzle_end_stroet} angestrebt, bei dem die Leerstelle die Nummer \afz{0} trägt.\\
%
\begin{minipage}{\linewidth}
	\begin{minipage}[t]{0.45\linewidth}
		\begin{figure}[H]
			\centering
			\includegraphics[width=\linewidth,keepaspectratio]{img/End_Puzzle_AO.png}
			\captionsetup{format=plain, indention=0pt}
			\caption{Häufig verwendeter Zielzustand eines 4x4-Puzzles \label{fig:Perm_puzzle_end_allOther}}
		\end{figure}
	\end{minipage}
	\hfill
	\begin{minipage}[t]{0.45\linewidth}
		\begin{figure}[H]
			\centering
			\includegraphics[width=\linewidth,keepaspectratio]{img/End_Puzzle_Stroetmann.png}
			\captionsetup{format=plain, indention=0pt}
			\caption{\label{fig:Perm_puzzle_end_stroet}Verwendeter Zielzustand eines 4x4-Puzzles aus dem Skript von Herrn Stroetmann \autocite{github-stroetmann:online}}
		\end{figure}
	\end{minipage}
\end{minipage}\WNL%

\section{Permutationen} % (fold)
\label{cha:Permutationen}
% chapter Permutationen (end)
\section{Kontext Vorlesung + Abgrenzung} % (fold)
\label{cha:Kontext Vorlesung}

% chapter Kontext Vorlesung (end)
\section{Sortieralgorithmen} % (fold)
\label{cha:Sortieralgorithmen}
Mit Ordnung?!
% chapter Sortieralgorithmen (end)
% Endzustände sind nicht ineinander überführbar
% Betrachtung von 2x2 Puzzle um zu zeigen dass es immer 2pfade gibt
% Betrachtung des Loyd Puzzles
% Betrachtung der einen Vorgegebenen Lösung von Herrn Stroetmann 




% !TEX root =  master.tex
\chapter{Implementation}
\section{Umsetzung} % (fold)
\label{cha:Umsetzung}

% chapter Umsetzung (end)
\section{Testing} % (fold)
\label{cha:Testing}
Um nun verschiedene Puzzleinstanzen und deren L"osbarkeit anhand des Codes zu testen, werden verschiedene Startzust"ande und zu Testzwecken verschiedene Endzust"ande definiert. Anschlie"send werden alle Startzust"ande auf L"osbarkeit des \enquote{normalen} Endzustandes getestet.\\
Um eine verbose Ausgabe zu erm"oglichen, bei der noch die verschiedenen zuvor berechneten Werte ausgegeben werden k"onnen, kann der Funktion, welche die Puzzleinstanzen auf ihre L"osbarkeit pr"uft, noch eine entsprechende verbose Flag mitgegeben werden. Somit kann anhand von bekannten l"osbaren Instanzen "uberpr"uft werden, ob die Implementierung vollst"andig ist.\\
Der Output dieses Testes befindet sich im dazugeh"origen \textcolor{violet}{\href{https://github.com/stubifox/ai-termpaper/blob/main/code/15-solvable-v1.ipynb}{Jupyter-Notebook}}.\\
Diese Puzzle werden dann bspw. in das schon aus der Vorlesung bekannte \textcolor{violet}{\href{https://github.com/karlstroetmann/Artificial-Intelligence/blob/master/Python/1\%20Search/Iterative-Deepening-A-Star-Search.ipynb}{Jupyter-Notebook}} zum L"osen der Instanzen gegeben, so dass geschaut werden kann, ob diese in angemessener Zeit l"osbar sind, und wieviele Z"uge f"ur die jeweilige L"osung gebraucht werden. \\
Ein Screenshot des Outputs findet sich im Anhang unter \ref{app:fig:testing-output}.
Das Problem hierbei ist, dass nicht jedes als l"osbar einkategorisierte Puzzle auch in angemessener Zeit mit Hilfe des \textit{IDA*} oder \textit{A*} Algorithmus gel"ost werden kann. So ist es durchaus vorgekommen, dass die Schritte zur L"osbarkeit eines Puzzles mehrere Stunden dauern (vgl. \ref{app:fig:1h}), da vorher nicht bekannt ist, wieviele Schritte zur L"osung f"uhren.\\
Somit wei"s man initial nicht, ob das Puzzle richtig einklassifiziert wurde.
% chapter Testing (end)

% Fazit und Ausblick
% !TEX root =  master.tex
\chapter{Zusammenfassung}
Diese Arbeit zeigt, dass Instanzen des 15-Puzzle programmatisch auf ihre L"osbarkeit "uberpr"uft werden k"onnen, indem sie als Permutationen in einer sortierten Liste betrachtet werden.
Der daraus resultierende von Bradlow vorgestellte Algorithmus ist leicht zu verstehen und l"asst sich ohne Umwege in einem Python-Modul abbilden. Dieses kann als Filter genutzt werden, um sicherzustellen, dass es eine L"osung gibt, die die in der Vorlesung behandeltelten Suchalgorithmen \cite[vgl.][Kap. 2]{github-stroetmann:online} finden  k"onnen.\\
Die "Uberpr"ufung durch einen Suchalgorithmus, der alle m"oglichen Zust"ande durcharbeitet ist mit einem h"oherem Zeitaufwand verbunden, als es der Filter ben"otigt.
So kann fr"uhzeitig erkannt werden, dass ein Puzzle nicht l"osbar ist und weder Suchalgorithmen noch Menschen wie in der Einleitung \ref{cha:Geschichte} m"ussen n"achtelang versuchen eine unl"osbare Aufgabe zu l"osen.
\WNL
Zuk"unftig k"onnte das Modul so erweitert werden, dass beliebige Endzust"ande m"oglich sind, wie bspw. in \ref{code:validate-15-puzzle:py}(Z. 52 - Z. 79) definiert sind. Aktulel wird nur auf die L"osbarkeit mit dem von Herrn Stroetmann definierten Endzustand gepr"uft
%%%%%%%%%%%%%%%%%%%%%%%%%%%%%%%%%%%


%%%%%%%%%%%%%%%%%%%%%%%%%%%%%%%%%%%
% LITERATURVERZEICHNIS
% 
% @stud: Literaturverzeichnis in Datei bibliography.bib anpassen 
%
\initializeBibliography
%%%%%%%%%%%%%%%%%%%%%%%%%%%%%%%%%%%



\initializeAppendix

%%%%%%%%%%%%%%%%%%%%%%%%%%%%%%%%%%%
% ANHÄNGE
%
% @stud: einzelne Anhänge bearbeiten und eigene Anhänge hier einfügen 
%
% !TEX root =  master.tex
\chapter{Anhang}


\subsubsection{Python Code zur Validierung der L"osbarkeit von Instanzen des 15-Puzzles}
\label{ssec:appendix-latency-benchmark}
\begin{lstlisting}[caption={Python Code zur Validierung der L"osbarkeit von Instanzen des 15-Puzzles}, label={code:validate-15-puzzle:py}]
\end{lstlisting}
\inputminted[linenos,breaklines,breakanywhere]{python}{../code/15-solvable.py}
\newpage

\subsubsection{Testing Output der verschiedenen Startzust\"ande}
\begin{figure}[H]
    \centering
    \includegraphics[width=\linewidth,keepaspectratio]{img/testing_output.PNG}
    \captionsetup{format=plain, indention=0pt}
    \caption{Testing Output der verschiedenen Startzust"ande \label{app:fig:testing-output}}
\end{figure}

\subsubsection{Puzzleinstanz mit hoher L"osbarkeitsberechnungsdauer}
\begin{figure}[H]
    \centering
    \includegraphics[width=0.5\linewidth,keepaspectratio]{img/time_consuming_puzzle_instance.png}
    \captionsetup{format=plain, indention=0pt}
    \caption{L"osbarkeitsberechnungsdauer \label{app:fig:time_c_p_i}}
\end{figure}

\subsubsection{Exemplarisches L"osen der in \ref{app:fig:time_c_p_i} angegebenen Puzzleinstanz (1h)}
\begin{figure}[H]
    \centering
    \includegraphics[width=\linewidth,keepaspectratio]{img/1h_solving_time_15_puzzle.png}
    \captionsetup{format=plain, indention=0pt}
    \caption{Exemplarisches L"osen eines 15-Puzzles mit hohem Zeitaufwand (1h) \label{app:fig:1h}}
\end{figure}
%%%%%%%%%%%%%%%%%%%%%%%%%%%%%%%%%%%

\singlespacing




%%%%%%%%%%%%%%%%%%%%%%%%%%%%%%%%%%%
% Inhdex
% 
% @stud: Hier kann ein Index erzeugt werden
%
%\addcontentsline{toc}{chapter}{Index}
%\printindex
%%%%%%%%%%%%%%%%%%%%%%%%%%%%%%%%%%%


\end{document}
