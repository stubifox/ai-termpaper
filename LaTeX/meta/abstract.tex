% !TEX root =  master.tex
\chapter*{Abstract}

The 15-puzzle first gained public attention in the 1870s when Sam Loyd promised $\$1000$ in prize money to anyone who could solve his puzzle, which became known as the 14-15 puzzle.\\
It was not until nearly a decade later that authors of the American Journal of Mathematics published a proof that the puzzle is unsolvable.
Today, the 15-puzzle is a classic use case when it comes to modelling algorithms with heuristics, such as \textit{A*} or \textit{IDA*}.\\
The main purpose of this work is to outline and implement the algorithm for checking the solvability of an instance of the 15-puzzle.
This algorithm is based on Bradlow's approach where a puzzle is considered solvable, if the parity of the transposition count and the manhattan distance of the blank field match.\\The In the process written source-code can be found within the appendix. To test this implementation, an evaluation of specific test-cases is conducted.\\
Altough the algorithm specifies the solvability for a given puzzle instance, it remains uncertain how many steps are required to obtain this solution. Hence the algorithm provides no indicator for computing time.
